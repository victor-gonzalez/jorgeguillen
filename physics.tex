\documentclass[14pt,aspectratio=169,t]{beamer}

\mode<presentation> {
%\usetheme{Hannover}

%\usecolortheme{albatross}
%\usecolortheme{beaver}
%\usecolortheme{beetle}
%\usecolortheme{crane}
%\usecolortheme{dolphin}
%\usecolortheme{dove}
%\usecolortheme{fly}
%\usecolortheme{lily}
%\usecolortheme{orchid}
\usecolortheme{rose}
%\usecolortheme{seagull}
%\usecolortheme{seahorse}
%\usecolortheme{whale}
%\usecolortheme{wolverine}

% To remove the footer line in all slides uncomment this line
%\setbeamertemplate{footline} 

% To replace the footer line in all
% slides with a simple slide count uncomment this line
%\setbeamertemplate{footline}[page number]
\setbeamertemplate{section in toc}[ball numbered] 

% To remove the navigation symbols
%from the bottom of all slides uncomment this line
\setbeamertemplate{navigation symbols}{} 
\usefonttheme{professionalfonts}


\setbeamertemplate{itemize items}[square] % if you want a square
\setbeamertemplate{itemize subitem}[circle] % if you wnat a circle
\setbeamertemplate{itemize subsubitem}[triangle] % if you want a triangle
}


\usepackage[utf8]{inputenc}
\usepackage{amsmath}
\usepackage{amsfonts}
\usepackage{amssymb}
\usepackage{amsthm}
\usepackage{slashed}

\usepackage{graphicx} % Allows including images
\usepackage{booktabs} % Allows \toprule, \midrule and \bottomrule in tables
\usepackage{caption}
\usepackage{subcaption}

\usepackage{pgfplots}
\usepgfplotslibrary{dateplot}
\pgfplotsset{compat=1.15}
\usepackage{xspace}

\usepackage{tikz}
%\usepackage{soul}
\usepackage[normalem]{ulem}

\usepackage[bigfiles]{pdfbase}
\usepackage[bigfiles]{media9}

\usepackage[overlay,absolute]{textpos}
\setlength{\TPHorizModule}{1cm}
\setlength{\TPVertModule}{1cm}
\textblockorigin{0mm}{0mm}


\makeatletter
\def\beamer@andinst{\quad}
\def\beamer@and{\\[1em]}
\makeatother


\definecolor{carmine}{rgb}{0.59, 0.0, 0.09}



\usepackage{graphicx}

\graphicspath{
  {./IMAGES/}
}

\title{\LARGE \textbf{La física\\Una aventura del conocimiento}}
\author{Víctor}
\date{}

% to insert the section title slide
\AtBeginSection[]{
  \begin{frame}
  \vfill
  \centering
  \begin{beamercolorbox}[sep=8pt,center,shadow=true,rounded=true]{title}
    \usebeamerfont{title}\insertsectionhead\par%
  \end{beamercolorbox}
  \vfill
  \end{frame}
}

%remove the icon
\setbeamertemplate{bibliography item}{}

\begin{document}

\begin{frame}{}
  \begin{textblock}{8}(7.0,4.0)
%      \titlepage
  {\usebeamerfont{title}\usebeamercolor[fg]{title}{ 
  \textbf{La Física\\Una aventura ...\\... del conocimiento}}}\\
  \vspace{0.1in}
  {\large \textbf{Víctor González}\\}
  {\scriptsize \textbf{Licenciado en Informática}\\}
  {\scriptsize \textbf{Doctor en Ciencias Físicas}\\}
%  \vspace{0.2in}
  {\small\textbf{20 de abril de 2023}}
  \end{textblock}
  
  % the logos
  \begin{textblock}{12}(1.0,0.3)
      \includegraphics[width=6cm]{LOGO-WSU-wsu_primary_horz_color}
  \end{textblock}
  \begin{textblock}{10}(12.0,0.2)
      \includegraphics[width=2.0cm]{CERN_logo-689x700}
  \end{textblock}
  \begin{textblock}{10}(14.3,0.2)
      \includegraphics[width=1.5cm]{LOGO-ALICE-2012-Jul-04-4_Color_Logo_CB}
  \end{textblock}
  \begin{textblock}{10}(0.2,2.9)
      \includegraphics[width=6.0cm]{jorgeguillenlogo}
  \end{textblock}
\end{frame}

\section{\LARGE \textbf{Las aventuras del conocimiento\\\vspace{.2in}-- La aventura humana --}}

\begin{frame}{}
  \only<1>{
    \begin{textblock}{16}(0.5,0.3)
      {\LARGE \usebeamercolor[fg]{title}\textbf{La gran pirámide de Giza}}
    \end{textblock}
    \begin{textblock}{14}(1.0,1.3)
      \centering
      \includegraphics[width=0.8\textwidth,keepaspectratio=true,
                      clip=true,trim=0pt 0pt 0pt 0pt]
                      {Kheops-Pyramid}\\
    \vspace{-0.1in}
    {\color{blue}\scriptsize By Nina - Own work, CC BY 2.5,
    \href{https://commons.wikimedia.org/w/index.php?curid=282496}
    {https://commons.wikimedia.org/w/index.php?curid=282496}}
    \end{textblock}
  }
  \only<2>{
    \begin{textblock}{16}(0.5,0.3)
      {\LARGE \usebeamercolor[fg]{title}\textbf{La Gran Muralla China}}
    \end{textblock}
    \begin{textblock}{14}(1.0,1.3)
      \centering
      \includegraphics[width=0.75\textwidth,keepaspectratio=true,
                      clip=true,trim=0pt 0pt 0pt 0pt]
                      {The_Great_Wall_of_China_at_Jinshanling-edit}\\
    \vspace{-0.1in}
    {\color{blue}\scriptsize By Severin.stalder, CC BY-SA 3.0,
    \href{https://commons.wikimedia.org/w/index.php?curid=39661035}
    {https://commons.wikimedia.org/w/index.php?curid=39661035}}
    \end{textblock}
  }
  \only<3>{
    \begin{textblock}{16}(0.5,0.3)
      {\LARGE \usebeamercolor[fg]{title}\textbf{Notre Dame de París}}
    \end{textblock}
    \begin{textblock}{14}(1.0,1.3)
      \centering
      \includegraphics[width=0.75\textwidth,keepaspectratio=true,
                      clip=true,trim=0pt 0pt 0pt 0pt]
                      {Notre-Dame_de_Paris,_4_October_2017}\\
    \vspace{-0.1in}
    {\color{blue}\scriptsize By Ali Sabbagh - Notre Dam De Paris, CC0,
    \href{https://commons.wikimedia.org/w/index.php?curid=84033306}
    {https://commons.wikimedia.org/w/index.php?curid=84033306}}
    \end{textblock}
  }
  \only<4>{
    \begin{textblock}{16}(0.5,0.3)
      {\LARGE \usebeamercolor[fg]{title}\textbf{San Lorenzo de El Escorial}}
    \end{textblock}
    \begin{textblock}{14}(1.0,1.3)
      \centering
      \includegraphics[width=0.66\textwidth,keepaspectratio=true,
                      clip=true,trim=0pt 0pt 0pt 0pt]
                      {1280px-El_Escorial,_Madrid,_Spain_(cropped)}\\
    \vspace{-0.0in}
    {\color{blue}\scriptsize De David Mapletoft - https://www.flickr.com/photos/imagine\_david/49812156266/, CC BY 2.0,\\
    \vspace{-0.12in}
    \href{https://commons.wikimedia.org/w/index.php?curid=89543317}
    {\tiny https://commons.wikimedia.org/w/index.php?curid=89543317}}
    \end{textblock}
  }
  \only<5>{
    \begin{textblock}{8}(0.5,0.3)
      {\LARGE \usebeamercolor[fg]{title}\textbf{David}}
    \end{textblock}
    \begin{textblock}{7}(1.0,1.3)
      \centering
      \includegraphics[width=0.635\textwidth,keepaspectratio=true,
                      clip=true,trim=0pt 0pt 0pt 0pt]
                      {Michelangelo's_David_-_right_view_2}\\
    \vspace{-0.00in}
    {\color{blue}\scriptsize By Commonists - Own work, CC BY-SA 4.0,\\
    \vspace{-0.12in}
    \href{https://commons.wikimedia.org/w/index.php?curid=110947601}
    {\tiny https://commons.wikimedia.org/w/index.php?curid=110947601}}
    \end{textblock}
    \begin{textblock}{8}(8.5,0.3)
      {\LARGE \usebeamercolor[fg]{title}\textbf{El pensador}}
    \end{textblock}
    \begin{textblock}{7}(8.0,1.3)
      \centering
      \includegraphics[width=0.62\textwidth,keepaspectratio=true,
                      clip=true,trim=0pt 0pt 0pt 0pt]
                      {Musée_Rodin_1}\\
    \vspace{-0.00in}
    {\color{blue}\scriptsize By CrisNYCa - Own work, CC BY-SA 4.0,\\
    \vspace{-0.12in}
    \href{https://commons.wikimedia.org/w/index.php?curid=94532194}
    {\tiny https://commons.wikimedia.org/w/index.php?curid=94532194}}
    \end{textblock}
  }
  \only<6>{
    \begin{textblock}{8}(0.5,0.3)
      {\LARGE \usebeamercolor[fg]{title}\textbf{Mona Lisa}}
    \end{textblock}
    \begin{textblock}{7}(1.0,1.3)
      \centering
      \includegraphics[width=0.66\textwidth,keepaspectratio=true,
                      clip=true,trim=0pt 0pt 0pt 0pt]
                      {Mona_Lisa,_by_Leonardo_da_Vinci,_from_C2RMF}\\
    \vspace{-0.1in}
    {\color{blue}\scalebox{0.8}{\scriptsize Leonardo da Vinci, Public domain, via Wikimedia Commons,}\\
    \vspace{-0.12in}
    \href{https://commons.wikimedia.org/wiki/File:Mona_Lisa,_by_Leonardo_da_Vinci,_from_C2RMF.jpg}
    {\tiny https://commons.wikimedia.org/wiki/File:Mona\_Lisa,\_by\_Leonardo\_da\_Vinci,\_from\_C2RMF.jpg}}
    \end{textblock}
    \begin{textblock}{8}(8.5,0.3)
      {\LARGE \usebeamercolor[fg]{title}\textbf{Inocencio X}}
    \end{textblock}
    \begin{textblock}{7}(8.0,1.3)
      \centering
      \includegraphics[width=0.7\textwidth,keepaspectratio=true,
                      clip=true,trim=0pt 0pt 0pt 0pt]
                      {Retrato_del_Papa_Inocencio_X._Roma,_by_Diego_Velázquez}\\
    \vspace{-0.00in}
    {\color{blue}\scriptsize De Diego Velázquez\\
    \vspace{-0.05in}
    \scalebox{0.6}{\tiny http://picasaweb.google.com/EnrikeCdC/VelazquezObraCompleta\#5297131513290212530,
    Dominio público,}\\
    \vspace{-0.12in}
    \href{https://commons.wikimedia.org/w/index.php?curid=9556265}
    {\tiny https://commons.wikimedia.org/w/index.php?curid=9556265}}
    \end{textblock}
  }
  \only<7>{
    \begin{textblock}{16}(0.5,0.3)
      {\LARGE \usebeamercolor[fg]{title}\textbf{Guernica}}
    \end{textblock}
    \begin{textblock}{14}(1.0,1.3)
      \centering
      \includegraphics[width=1\textwidth,keepaspectratio=true,
                      clip=true,trim=0pt 0pt 0pt 0pt]
                      {GuernikaPabloPicasso}\\
    \vspace{-0.0in}
    {\color{blue}\scriptsize Museo Nacional. Centro de Arte Reina Sofía\\
    \vspace{-0.12in}
    \href{https://www.museoreinasofia.es/en/collection/artwork/guernica}
    {\tiny https://www.museoreinasofia.es/en/collection/artwork/guernica}}
    \end{textblock}
  }
  \only<8>{
    \begin{textblock}{3}(0.2,0.3)
      \centering
      \includegraphics[width=1.0\textwidth,keepaspectratio=true,
                      clip=true,trim=0pt 0pt 0pt 0pt]
                      {Title_page_first_edition_Don_Quijote}
    \end{textblock}
    \begin{textblock}{3}(3.2,0.5)
      \centering
      \includegraphics[width=1.0\textwidth,keepaspectratio=true,
                      clip=true,trim=0pt 0pt 0pt 0pt]
                      {Lady_chatterley's_lover_1932_UK_(Secker)}
    \end{textblock}
    \begin{textblock}{3}(6.5,0.7)
      \centering
      \includegraphics[width=1.0\textwidth,keepaspectratio=true,
                      clip=true,trim=0pt 0pt 0pt 0pt]
                      {Neruda_-_Portada_Veinte_poemas_de_Amor_(1924)}
    \end{textblock}
    \begin{textblock}{3}(9.7,0.8)
      \centering
      \includegraphics[width=1.0\textwidth,keepaspectratio=true,
                      clip=true,trim=0pt 0pt 0pt 0pt]
                      {JoyceUlysses2}
    \end{textblock}
    \begin{textblock}{3}(12.9,0.65)
      \centering
      \includegraphics[width=1.0\textwidth,keepaspectratio=true,
                      clip=true,trim=0pt 0pt 0pt 0pt]
                      {Cien_años_de_soledad}
    \end{textblock}
    \begin{textblock}{14}(1.0,5.5)
    \vspace{-0.0in}
    {\color{blue}\scriptsize De Juan de la Cuesta
    \scalebox{0.5}{- https://legendsrevealed.com/entertainment/2012/08/08/did-cervantes-write-don-quixote-while-in-prison/,} Dominio público\\
    \vspace{-0.12in}
    \href{https://commons.wikimedia.org/w/index.php?curid=108839792}
    {\tiny https://commons.wikimedia.org/w/index.php?curid=108839792}}

    \vspace{-0.0in}
    {\color{blue}\scriptsize By May be found at the following website: Abebooks
    \scalebox{0.5}{see also other listings of the 1932 edition in Abebooks and other shopping websites,} Fair use,\\
    \vspace{-0.12in}
    \href{https://en.wikipedia.org/w/index.php?curid=65746342}
    {\tiny https://en.wikipedia.org/w/index.php?curid=65746342}}

    \vspace{-0.0in}
    {\color{blue}\scriptsize De Santiago: Editorial Nascimento.
    \scalebox{0.5}{- http://www.memoriachilena.cl/602/w3-article-100051.html,} Dominio público,\\
    \vspace{-0.12in}
    \href{https://commons.wikimedia.org/w/index.php?curid=55262392}
    {\tiny https://commons.wikimedia.org/w/index.php?curid=55262392}}

    \vspace{-0.0in}
    {\color{blue}\scriptsize By James Joyce - Own work, Public Domain,\\
    \vspace{-0.12in}
    \href{https://commons.wikimedia.org/w/index.php?curid=555052}
    {\tiny https://commons.wikimedia.org/w/index.php?curid=555052}}

    \vspace{-0.0in}
    {\color{blue}\scriptsize De Luis Rivera - Cien años de Soledad - Edición de 2017 - Editorial Random, Dominio público\\
    \vspace{-0.12in}
    \href{https://commons.wikimedia.org/w/index.php?curid=85847487}
    {\tiny https://commons.wikimedia.org/w/index.php?curid=85847487}}

    \end{textblock}
  }
  \only<9>{
    \begin{textblock}{3}(0.2,0.4)
      \centering
      {\color{blue}\Large \textbf{Bach}}\\
      \vspace{0.05in}
      \includegraphics[width=1.0\textwidth,keepaspectratio=true,
                      clip=true,trim=0pt 0pt 0pt 0pt]
                      {Johann_Sebastian_Bach}
    \end{textblock}
    \begin{textblock}{4}(2.9,0.4)
      \centering
      {\color{blue}\Large \textbf{\hspace{-0.00in}Beethoven}}\\
      \vspace{0.05in}
      \includegraphics[width=0.75\textwidth,keepaspectratio=true,
                      clip=true,trim=0pt 0pt 0pt 0pt]
                      {Beethoven}
    \end{textblock}
    \begin{textblock}{5.8}(6.6,0.4)
      \centering
      {\color{blue}\Large \textbf{\hspace{-0.00in}Deep Purple}}\\
      \vspace{0.05in}
      \includegraphics[width=1.0\textwidth,keepaspectratio=true,
                      clip=true,trim=0pt 0pt 0pt 0pt]
                      {Deeppurple_poster_pelo_1972}
    \end{textblock}
    \begin{textblock}{3.4}(12.5,0.6)
      \centering
      \includegraphics[width=1.0\textwidth,keepaspectratio=true,
                      clip=true,trim=0pt 0pt 0pt 0pt]
                      {Queen-1977ADATRT}
    \end{textblock}
    \begin{textblock}{14}(1.0,6.0)
    \vspace{-0.0in}
    {\color{blue}\scriptsize By Elias Gottlob Haussmann
    \scalebox{1.0}{- http://www.jsbach.net/bass/elements/bach-hausmann.jpg,} Public Domain\\
    \vspace{-0.12in}
    \href{https://commons.wikimedia.org/w/index.php?curid=1270015}
    {\tiny https://commons.wikimedia.org/w/index.php?curid=1270015}}

    \vspace{-0.0in}
    {\color{blue}\scriptsize By Karl Joseph Stieler - web.archive.org/web/20160623080009, Public Domain\\
    \vspace{-0.12in}
    \href{https://commons.wikimedia.org/w/index.php?curid=165990}
    {\tiny https://commons.wikimedia.org/w/index.php?curid=165990}}

    \vspace{-0.0in}
    {\color{blue}\scriptsize By Unknown author - Pelo magazine \#33, 1972, Public Domain\\
    \vspace{-0.12in}
    \href{https://commons.wikimedia.org/w/index.php?curid=114025311}
    {\tiny https://commons.wikimedia.org/w/index.php?curid=114025311}}

    \vspace{-0.0in}
    {\color{blue}\scriptsize By TheMillionaireWaltz - Own work, CC BY-SA 4.0\\
    \vspace{-0.12in}
    \href{https://commons.wikimedia.org/w/index.php?curid=78092381}
    {\tiny https://commons.wikimedia.org/w/index.php?curid=78092381}}
    \end{textblock}
  }
  \only<10>{
    \begin{textblock}{4}(0.2,0.4)
      \centering
      \includegraphics[width=1.0\textwidth,keepaspectratio=true,
                      clip=true,trim=0pt 0pt 0pt 0pt]
                      {Argentina_vs_mexico_messi_vs_alvarado}
    \end{textblock}
    \begin{textblock}{4.6}(4.4,0.6)
      \vspace{0.05in}
      \includegraphics[width=0.9\textwidth,keepaspectratio=true,
                      clip=true,trim=0pt 0pt 240pt 0pt]
                      {Rafael_Nadal_(12054444625)}
    \end{textblock}
    \begin{textblock}{7.2}(8.7,0.4)
      \vspace{0.05in}
      \includegraphics[width=1.0\textwidth,keepaspectratio=true,
                      clip=true,trim=0pt 0pt 40pt 0pt]
                      {Carolin_Marin}
    \end{textblock}
    \begin{textblock}{14}(1.0,6.8)
    \vspace{-0.0in}
    {\color{blue}\scriptsize De Tasnim News Agency, CC BY 4.0\\
    \vspace{-0.12in}
    \href{https://commons.wikimedia.org/w/index.php?curid=126794572}
    {\tiny https://commons.wikimedia.org/w/index.php?curid=126794572}}

    \vspace{-0.0in}
    {\color{blue}\scriptsize De Doha Stadium Plus Qatar/ Vinod Divakaran
    \scalebox{0.5}{- https://www.flickr.com/photos/dohastadiumplusqatar/12054444625/,} CC BY 2.0\\
    \vspace{-0.12in}
    \href{https://commons.wikimedia.org/w/index.php?curid=88966888}
    {\tiny https://commons.wikimedia.org/w/index.php?curid=88966888}}

    \vspace{-0.0in}
    {\color{blue}\scriptsize By TSportsasia - Own work, CC BY-SA 3.0\\
    \vspace{-0.12in}
    \href{https://commons.wikimedia.org/w/index.php?curid=32168875}
    {\tiny https://commons.wikimedia.org/w/index.php?curid=32168875}}
    \end{textblock}
  }
  \only<11>{
    \begin{textblock}{8}(0.1,0.7)
      \centering
      \includegraphics[width=0.90\textwidth,keepaspectratio=true,
                      clip=true,trim=0pt 0pt 0pt 0pt]
                      {Aldrin_Apollo_11_original}\\
    \vspace{-0.1in}
    {\color{blue} \scalebox{0.8}{\scriptsize By Neil Armstrong
    \scalebox{0.5}{\scriptsize - http://www.hq.nasa.gov/alsj/a11/AS11-40-5903HR.jpg NASA Image and Video Library,} Public Domain}\\
    \vspace{-0.12in}
    \href{https://commons.wikimedia.org/w/index.php?curid=3847764}
    {\tiny https://commons.wikimedia.org/w/index.php?curid=3847764}}
    \end{textblock}
    \begin{textblock}{8}(8.0,0.3)
      \centering
      \includegraphics[width=1.0\textwidth,keepaspectratio=true,
                      clip=true,trim=60pt 0pt 0pt 0pt]
                      {JWST_spacecraft_model_3}\\
    \vspace{-0.00in}
    {\color{blue}\scriptsize By NASA
    \scalebox{0.5}{\scriptsize - https://www.jwst.nasa.gov/content/webbLaunch/whereIsWebb.html,} Public Domain\\
    \vspace{-0.12in}
    \href{https://commons.wikimedia.org/w/index.php?curid=114125719}
    {\tiny https://commons.wikimedia.org/w/index.php?curid=114125719}}
    \end{textblock}
  }
\end{frame}

\section{\LARGE \textbf{La Física}}

\begin{frame}
  \frametitle{\LARGE \textbf{La contribución de la Física}}
  \begin{textblock}{15}(0.5,1.5)
    \includegraphics[width=0.49\textwidth,keepaspectratio=true,
                     clip=true,trim=20pt 0pt 0pt 0pt]
                     {pedaa5b25f02_hr}
    \includegraphics[width=0.49\textwidth,keepaspectratio=true,
                     clip=true,trim=0pt 100pt 0pt 0pt]
                     {s-l500}
  \end{textblock}

\end{frame}

\begin{frame}
  \frametitle{\LARGE \textbf{Teoría de la Relatividad}}
  \begin{textblock}{7.35}(8.0,1.5)
    \includegraphics[width=1\textwidth,keepaspectratio=true,
                     clip=true,trim=0pt 100pt 0pt 0pt]
                     {s-l500}
  \end{textblock}
  \begin{textblock}{7.35}(0.5,2.0)
    \begin{itemize}
      \item {\color{blue}\textbf{Luz no instantánea}}
      \vspace{0.05in}
      \item {\color{blue}\textbf{Espaciotiempo}}
      \vspace{0.05in}
      \item {\color{blue}\textbf{Dilatación del tiempo}}
      \vspace{0.05in}
      \item {\color{blue}\textbf{Agujeros negros}}
      \vspace{0.05in}
      \item {\color{blue}\textbf{Materia oscura}}
      \vspace{0.05in}
      \item {\color{blue}\textbf{Energía oscura}}
    \end{itemize}
  \end{textblock}
\end{frame}

\begin{frame}
  \frametitle{\LARGE \textbf{}}
  \begin{textblock}{15}(0.5,0.0)
    \centering
    \includegraphics[width=0.84\textwidth,keepaspectratio=true,
                     clip=true,trim=0pt 0pt 0pt 0pt]
                     {1620px-Cosmic_Calendar}\\
    \vspace{-0.1in}
    {\color{blue}\scriptsize By Efbrazil, CC BY-SA 3.0
    \href{https://commons.wikimedia.org/w/index.php?curid=18385338}
    {\tiny https://commons.wikimedia.org/w/index.php?curid=18385338}}
  \end{textblock}
\end{frame}


\begin{frame}
  \frametitle{\LARGE \textbf{Sabemos sólo acerca de un 5\%}}
  \begin{textblock}{8}(0.5,1.5)
    \includegraphics[width=0.9\textwidth,keepaspectratio=true,
                     clip=true,trim=0pt 0pt 0pt 0pt]
                     {INK/DarkEnergy}
  \end{textblock}
  \begin{textblock}{9}(6.7,1.5)
    \centering
    {\color{blue}\textbf{\large Los brazos de las galaxias giran más rápido de lo esperado}}
    \includegraphics[width=0.6\textwidth,keepaspectratio=true,
                     clip=true,trim=0pt 0pt 0pt 0pt]
                     {INK/GalacticRotation2}\\
    {\color{blue}\scriptsize By PhilHibbs - Own work in Inkscape 0.42, Public Domain,\\
    \href{https://commons.wikimedia.org/w/index.php?curid=365013}
    {\tiny https://commons.wikimedia.org/w/index.php?curid=365013}\\}
  \end{textblock}
  \begin{textblock}{15}(0.5,7.0)
    {\color{blue}\textbf{\large Las supenovas {\LARGE Ia} muestran que el Universo\\
        se expande de forma acelerada} (Nóbel de Física 2011)}
  \end{textblock}
\end{frame}

\section{\LARGE \textbf{¿Se transmite la luz\\instantáneamente?\\¿Cómo se mide\\la velocidad de la luz?}}

\begin{frame}
  \frametitle{{\LARGE\textbf{IO.}}{\Large\textbf{ La luna más interna de Júpiter}}}
  \begin{textblock}{15}(0.5,1.2)
    \centering
    \includegraphics[width=0.25\textwidth,keepaspectratio=true,
                     clip=true,trim=0pt 0pt 0pt 10pt]
                     {IO_one}
    \hspace{0.4in}%
    \includegraphics[width=0.25\textwidth,keepaspectratio=true,
                     clip=true,trim=0pt 0pt 0pt 0pt]
                     {IO_two}\\
    \vspace{-0.1in}
    {\color{blue}\scriptsize
    \href{https://en.wikipedia.org/wiki/Io_(moon)}
    {https://en.wikipedia.org/wiki/Io\_(moon)}}
  \end{textblock}
\end{frame}

\begin{frame}
  \frametitle{\LARGE\textbf{Ole Christensen Rømer(1676)}}
  \begin{textblock}{14}(1.0,1.3)
    \centering
    \includegraphics[width=1\textwidth,keepaspectratio=true,
                     clip=true,trim=0pt 0pt 0pt 0pt]
                     {Io_eclipse_speed_of_light_measurement}\\
    \vspace{-0.1in}
    {\color{blue}\scriptsize By Cmglee - Own work, CC BY-SA 3.0
    \href{https://commons.wikimedia.org/w/index.php?curid=32388964}
    {https://commons.wikimedia.org/w/index.php?curid=32388964}}
  \end{textblock}
\end{frame}


\begin{frame}
  \frametitle{\LARGE \textbf{El Modelo Estándar}}
  \begin{textblock}{7.35}(0.5,1.5)
    \includegraphics[width=1\textwidth,keepaspectratio=true,
                     clip=true,trim=20pt 0pt 0pt 0pt]
                     {pedaa5b25f02_hr}
  \end{textblock}
  \begin{textblock}{7.00}(8.3,2.0)
    \begin{itemize}
      \item {\color{blue}\textbf{La partículas son ondas}}
      \vspace{0.05in}
      \item {\color{blue}\textbf{Las ondas son partículas}}
      \vspace{0.05in}
      \item {\color{blue}\textbf{Familias de parículas fundamentales}}
      \vspace{0.05in}
      \item {\color{blue}\textbf{Tres fuerzas fundamentales}}
    \end{itemize}
  \end{textblock}
\end{frame}

\section{\LARGE\textbf{¿Qué son las ondas?}}

\begin{frame}{The waves video}
%  \begin{textblock}{16.05}(-0.02,-0.02)
%  \includemedia[
%    width=1\linewidth,
%    height=0.565\linewidth,
%    keepaspectratio,
%    activate=pageopen,
%    playbutton=fancy,
%    addresource=VIDEOS/waves.mp4,
%    flashvars={source=VIDEOS/waves.mp4&autoPlay=true&loop=true}
%    ]{}{VPlayer.swf}
%  \end{textblock}
  \href{https://www.youtube.com/watch?v=Jqm4f55soJQ}
      {\color{blue}\underline{https://www.youtube.com/watch?v=Jqm4f55soJQ}}
\end{frame}

\section{\LARGE\textbf{¿Cómo se comporta la luz?\\Aparentemente}}

\begin{frame}{The light double slit video}
%  \begin{textblock}{16.05}(-0.02,-0.02)
%  \includemedia[
%    width=1\linewidth,
%    height=0.565\linewidth,
%    keepaspectratio,
%    activate=pageopen,
%    playbutton=fancy,
%    addresource=VIDEOS/lightdslit.mp4,
%    flashvars={source=VIDEOS/lightdslit.mp4&autoPlay=true&loop=true}
%    ]{}{VPlayer.swf}
%  \end{textblock}
  \href{https://www.youtube.com/watch?v=x2upztBHo7o}
      {\color{blue}\underline{https://www.youtube.com/watch?v=x2upztBHo7o}}
\end{frame}

\section{\LARGE\textbf{¿Cómo esperamos\\que se comporten las partículas?}}

\begin{frame}{The particles double slit video}
%  \begin{textblock}{16.05}(-0.02,-0.02)
%  \includemedia[
%    width=1\linewidth,
%    height=0.565\linewidth,
%    keepaspectratio,
%    activate=pageopen,
%    playbutton=fancy,
%    addresource=VIDEOS/ballsdslit.mp4,
%    flashvars={source=VIDEOS/ballsdslit.mp4&autoPlay=true&loop=true}
%    ]{}{VPlayer.swf}
%  \end{textblock}
  \href{https://www.youtube.com/watch?v=1tROUlC6IL8}
      {\color{blue}\underline{https://www.youtube.com/watch?v=1tROUlC6IL8}}
\end{frame}

\section{\LARGE\textbf{¿Cómo se comporta la naturaleza?\\Luz y\\partículas}}

\begin{frame}{The nature double slit video}
%  \begin{textblock}{16.05}(-0.02,-0.02)
%  \includemedia[
%    width=1\linewidth,
%    height=0.565\linewidth,
%    keepaspectratio,
%    activate=pageopen,
%    playbutton=fancy,
%    addresource=VIDEOS/measuredslit.mp4,
%    flashvars={source=VIDEOS/measuredslit.mp4&autoPlay=true&loop=true}
%    ]{}{VPlayer.swf}
%  \end{textblock}
  \href{https://www.youtube.com/watch?v=81kW0BUKHm4}
      {\color{blue}\underline{https://www.youtube.com/watch?v=81kW0BUKHm4}}
\end{frame}


\section{\LARGE\textbf{Partículas elementales}}

\begin{frame}
  \frametitle{\LARGE \textbf{Partículas y fuerzas elementales}}
  \begin{textblock}{15}(0.5,1.0)
    \centering
    \includegraphics[width=0.6\textwidth,keepaspectratio=true,
                     clip=true,trim=0pt 0pt 0pt 90pt]
                     {Standard_Model_of_Elementary_Particles}
  \end{textblock}
\end{frame}


\begin{frame}
  \frametitle{\LARGE \textbf{La evolución del Universo}}
  \begin{textblock}{15}(0.5,1.0)
    \centering
    \includegraphics[width=0.65\textwidth,keepaspectratio=true,
                     clip=true,trim=0pt 0pt 0pt 0pt]
                     {univ}\\
  \end{textblock}
\end{frame}

\begin{frame}
  \frametitle{\LARGE \textbf{¿Cómo se investiga?}}
  \begin{textblock}{15}(0.5,1.3)
    \centering
    \includegraphics[width=0.56\textwidth,keepaspectratio=true,
                     clip=true,trim=0pt 0pt 0pt 0pt]
                     {Zuniverse_original}\\
  \end{textblock}
\end{frame}

\section{\LARGE\textbf{El CERN}}

\begin{frame}{The CERN video}
%  \begin{textblock}{16.05}(-0.02,-0.02)
%  \includemedia[
%    width=1\linewidth,
%    height=0.565\linewidth,
%    keepaspectratio,
%    activate=pageopen,
%    playbutton=fancy,
%    addresource=VIDEOS/CERN-VIDEO-2019-019-001-1080p.mp4,
%    flashvars={source=VIDEOS/CERN-VIDEO-2019-019-001-1080p.mp4&autoPlay=true}
%    ]{}{VPlayer.swf}
%  \end{textblock}
  \href{https://home.cern/resources/video/cern/what-cern}
      {\color{blue}\underline{https://home.cern/resources/video/cern/what-cern}}
\end{frame}

\begin{frame}
  \frametitle{\LARGE \textbf{}}
  \begin{textblock}{15}(0.5,0.0)
    \centering
    \includegraphics[width=0.7\textwidth,keepaspectratio=true,
                     clip=true,trim=0pt 0pt 0pt 0pt]
                     {CCC-v2022_large}\\
  \end{textblock}
\end{frame}

\begin{frame}{The LHC CCC video}
%  \begin{textblock}{16.05}(-0.02,-0.02)
%  \includemedia[
%    width=1\linewidth,
%    height=0.565\linewidth,
%    keepaspectratio,
%    activate=pageopen,
%    playbutton=fancy,
%    addresource=VIDEOS/Acelerators.mp4,
%    flashvars={source=VIDEOS/Acelerators.mp4&autoPlay=true}
%    ]{}{VPlayer.swf}
%  \end{textblock}
  \href{https://home.cern/resources/video/computing/cern-control-centre-ccc-english}
      {\small\color{blue}\underline{https://home.cern/resources/video/computing/cern-control-centre-ccc-english}}
\end{frame}

\begin{frame}
  \frametitle{\LARGE \textbf{}}
  \begin{textblock}{15}(0.5,0.0)
    \centering
    \includegraphics[width=0.7\textwidth,keepaspectratio=true,
                     clip=true,trim=0pt 0pt 0pt 0pt]
                     {CCC-v2022_large}\\
  \end{textblock}
\end{frame}

\begin{frame}
  \frametitle{\LARGE \textbf{ALICE}}
  \framesubtitle{\large \textbf{A Large Ion Collider Experiment}}
  \begin{textblock}{15}(0.5,1.8)
    \includegraphics[width=1.0\textwidth,keepaspectratio=true,
                     clip=true,trim=0pt 0pt 0pt 0pt]
                     {2017-May-11-ALICE_RUN3_labels_HD}\\
  \end{textblock}
\end{frame}

\begin{frame}{The ALICE video}
%  \begin{textblock}{16.1}(0.0,0.0)
%  \includemedia[
%    width=1\linewidth,
%    height=0.56\linewidth,
%    keepaspectratio,
%    activate=pageopen,
%    playbutton=fancy,
%    addresource=VIDEOS/OPEN-VIDEO-2021-344-001.mp4,
%    flashvars={source=VIDEOS/OPEN-VIDEO-2021-344-001.mp4&autoPlay=true}
%    ]{}{VPlayer.swf}
%  \end{textblock}
  \href{https://home.cern/resources/video/experiments/alice-general-video}
      {\color{blue}\underline{https://home.cern/resources/video/experiments/alice-general-video}}
\end{frame}

\begin{frame}
  \frametitle{\LARGE \textbf{ALICE}}
  \framesubtitle{\large \textbf{A Large Ion Collider Experiment}}
  \begin{textblock}{15}(0.5,1.8)
    \includegraphics[width=1.0\textwidth,keepaspectratio=true,
                     clip=true,trim=0pt 0pt 0pt 0pt]
                     {2017-May-11-ALICE_RUN3_labels_HD}\\
  \end{textblock}
\end{frame}

\begin{frame}
  \frametitle{\LARGE \textbf{Una colisión pp}}
  \begin{textblock}{15}(0.5,1.2)
    \centering
    \includegraphics[width=0.87\textwidth,keepaspectratio=true,
                     clip=true,trim=0pt 0pt 0pt 0pt]
                     {run285602_ev217}\\
  \end{textblock}
\end{frame}


\begin{frame}
  \frametitle{\LARGE \textbf{Una colisión Pb--Pb}}
  \begin{textblock}{15}(0.5,1.2)
    \centering
    \includegraphics[width=0.87\textwidth,keepaspectratio=true,
                     clip=true,trim=0pt 0pt 0pt 0pt]
                     {ALICE_PbPb_collision}\\
  \end{textblock}
\end{frame}

\begin{frame}
  \frametitle{\LARGE \textbf{Un candidato a $\pi^{\rm o}$}}
  \begin{textblock}{15}(0.5,1.2)
    \centering
    \includegraphics[width=0.67\textwidth,keepaspectratio=true,
                     clip=true,trim=0pt 0pt 0pt 0pt]
                     {2012-May-29-Eventdisplay}\\
  \end{textblock}
\end{frame}

\begin{frame}

\end{frame}

\section{\LARGE \textbf{Cosmos}}

\begin{frame}
  \frametitle{\LARGE \textbf{Voyager 1 -- Carl Sagan}}
  \only<1>{
    \begin{textblock}{10}(0.5,1.2)
      \includegraphics[width=0.9\textwidth,keepaspectratio=true,
                      clip=true,trim=0pt 0pt 0pt 0pt]
                      {GIF/AnimatedVoyager-0}\\
    \end{textblock}
  }
  \only<2>{
    \begin{textblock}{10}(0.5,1.2)
      \includegraphics[width=0.9\textwidth,keepaspectratio=true,
                      clip=true,trim=0pt 0pt 0pt 0pt]
                      {GIF/AnimatedVoyager-1}\\
    \end{textblock}
  }
  \only<3>{
    \begin{textblock}{10}(0.5,1.2)
      \includegraphics[width=0.65\textwidth,keepaspectratio=true,
                      clip=true,trim=0pt 0pt 0pt 0pt]
                      {PIA23645}\\
    \end{textblock}
  }
  \begin{textblock}{7}(9.5,1.2)
    \centering
    \includegraphics[width=0.55\textwidth,keepaspectratio=true,
                    clip=true,trim=0pt 0pt 0pt 0pt]
                    {Carl_Sagan_-_1980}\\
    {\color{blue}\scriptsize By PBS - ebay, PD-US,\\
    \href{https://en.wikipedia.org/w/index.php?curid=42222590}
    {\tiny https://en.wikipedia.org/w/index.php?curid=42222590}\\}
  \end{textblock}
  \only<3>{
    \begin{textblock}{3.7}(7.2,1.2)
      \centering
      {\scriptsize\color{blue}
        \textbf{El 13 de febrero de 1990, a 6.000 millones de kilómetros, la Voyager 1 calentó sus
      cámaras durante tres horas.
      La plataforma científica espacial fue entonces enfocada hacia Neptuno y las observaciones comenzaron.
      Después de Neptuno tomó imágenes de Urano, Saturno, Marte y El Sol. Después, de Júpiter,
      La Tierra y Venus. Las imágenes de La Tierra se tomaron a las 04:48 GMT del dia 14 de febrero de 1990,
      justo 34 minutos antes de que la Voyager 1 desconectara sus cámaras para siempre.}\\
      }
    \end{textblock}
  }

  \begin{textblock}{15}(0.5,7.9)
  {\footnotesize\color{blue}
    \textbf{-- Voyager 1 y 2 fueron lanzadas en verano de 1977}

    \textbf{-- En agosto de 2012 y noviembre de 2018 entraron
    en el espacio interestelar}\\
  }
  \end{textblock}
  \begin{textblock}{3}(4.8,7.0)
   {\color{white}\textbf{\textcopyright NASA}}
  \end{textblock}


\end{frame}


\begin{frame}{A Pale Blue Dot}
%  \begin{textblock}{16.1}(0.0,0.0)
%  \includemedia[
%    width=1\linewidth,
%    height=0.56\linewidth,
%    keepaspectratio,
%    activate=pageopen,
%    playbutton=fancy,
%    addresource=VIDEOS/palebluedot.mp4,
%    flashvars={source=VIDEOS/palebluedot.mp4&autoPlay=true}
%    ]{}{VPlayer.swf}
%  \end{textblock}
  \href{https://www.youtube.com/watch?v=GO5FwsblpT8}
      {\color{blue}\underline{https://www.youtube.com/watch?v=GO5FwsblpT8}}
\end{frame}

\section{\LARGE \textbf{El planeta\\y el futuro ...\\\vspace{0.15in}... os están esperando}}
\section{\LARGE \textbf{-- Gracias por vuestra atención --}}

\end{document}

\begin{frame}{The CERN video}
  \begin{textblock}{16.05}(-0.02,-0.02)
  \includemedia[
    width=1\linewidth,
    height=0.565\linewidth,
    keepaspectratio,
    activate=pageopen,
    playbutton=fancy,
    addresource=VIDEOS/CERN-VIDEO-2019-019-001-1080p.mp4,
    flashvars={source=VIDEOS/CERN-VIDEO-2019-019-001-1080p.mp4&autoPlay=true}
    ]{}{VPlayer.swf}
  \end{textblock}

\end{frame}

\begin{frame}{The LHC CCC video}
  \begin{textblock}{16.05}(-0.02,-0.02)
  \includemedia[
    width=1\linewidth,
    height=0.565\linewidth,
    keepaspectratio,
    activate=pageopen,
    playbutton=fancy,
    addresource=VIDEOS/Acelerators.mp4,
    flashvars={source=VIDEOS/Acelerators.mp4&autoPlay=true}
    ]{}{VPlayer.swf}
  \end{textblock}

\end{frame}

\begin{frame}{The ALICE video}
  \begin{textblock}{16.1}(0.0,0.0)
  \includemedia[
    width=1\linewidth,
    height=0.56\linewidth,
    keepaspectratio,
    activate=pageopen,
    playbutton=fancy,
    addresource=VIDEOS/OPEN-VIDEO-2021-344-001.mp4,
    flashvars={source=VIDEOS/OPEN-VIDEO-2021-344-001.mp4&autoPlay=true}
    ]{}{VPlayer.swf}
  \end{textblock}

\end{frame}

\begin{frame}{The Pale blue dot video}
  \begin{textblock}{16.1}(0.0,0.0)
  \includemedia[
    width=1\linewidth,
    height=0.56\linewidth,
    keepaspectratio,
    activate=pageopen,
    playbutton=fancy,
    addresource=VIDEOS/Pale-Blue-Dot-2014-version.mov,
    flashvars={source=VIDEOS/palebluedot.mp4&autoPlay=true}
    ]{}{VPlayer.swf}
  \end{textblock}

\end{frame}



\end{document}
