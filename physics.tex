\documentclass[14pt,aspectratio=169,t]{beamer}

\mode<presentation> {
%\usetheme{Hannover}

%\usecolortheme{albatross}
%\usecolortheme{beaver}
%\usecolortheme{beetle}
%\usecolortheme{crane}
%\usecolortheme{dolphin}
%\usecolortheme{dove}
%\usecolortheme{fly}
%\usecolortheme{lily}
%\usecolortheme{orchid}
\usecolortheme{rose}
%\usecolortheme{seagull}
%\usecolortheme{seahorse}
%\usecolortheme{whale}
%\usecolortheme{wolverine}

% To remove the footer line in all slides uncomment this line
%\setbeamertemplate{footline} 

% To replace the footer line in all
% slides with a simple slide count uncomment this line
\setbeamertemplate{footline}[page number] 
\setbeamertemplate{section in toc}[ball numbered] 

% To remove the navigation symbols
%from the bottom of all slides uncomment this line
\setbeamertemplate{navigation symbols}{} 
\usefonttheme{professionalfonts}


\setbeamertemplate{itemize items}[square] % if you want a square
\setbeamertemplate{itemize subitem}[circle] % if you wnat a circle
\setbeamertemplate{itemize subsubitem}[triangle] % if you want a triangle
}


\usepackage[utf8]{inputenc}
\usepackage{amsmath}
\usepackage{amsfonts}
\usepackage{amssymb}
\usepackage{amsthm}
\usepackage{slashed}

\usepackage{graphicx} % Allows including images
\usepackage{booktabs} % Allows \toprule, \midrule and \bottomrule in tables
\usepackage{caption}
\usepackage{subcaption}

\usepackage{pgfplots}
\usepgfplotslibrary{dateplot}
\pgfplotsset{compat=1.15}
\usepackage{xspace}

\usepackage{tikz}
%\usepackage{soul}
\usepackage[normalem]{ulem}

\usepackage[bigfiles]{pdfbase}
\usepackage[bigfiles]{media9}

\usepackage[overlay,absolute]{textpos}
\setlength{\TPHorizModule}{1cm}
\setlength{\TPVertModule}{1cm}
\textblockorigin{0mm}{0mm}


\makeatletter
\def\beamer@andinst{\quad}
\def\beamer@and{\\[1em]}
\makeatother


\definecolor{carmine}{rgb}{0.59, 0.0, 0.09}



\usepackage{graphicx}

\graphicspath{
  {./IMAGES/}
}

\title{\LARGE \textbf{La física\\Una aventura del conocimiento}}
\author{Víctor}
\date{}

% to insert the section title slide
\AtBeginSection[]{
  \begin{frame}
  \vfill
  \centering
  \begin{beamercolorbox}[sep=8pt,center,shadow=true,rounded=true]{title}
    \usebeamerfont{title}\insertsectionhead\par%
  \end{beamercolorbox}
  \vfill
  \end{frame}
}

%remove the icon
\setbeamertemplate{bibliography item}{}

\begin{document}

\begin{frame}{}
  \begin{textblock}{8}(7.0,4.4)
%      \titlepage
  {\usebeamerfont{title}\usebeamercolor[fg]{title}{ 
  \textbf{La ciencia\\Una aventura ...\\... del conocimiento}}}\\
  \vspace{0.1in}
  {\large \textbf{Víctor}}\\
  \vspace{0.2in}
  {\small\textbf{20 de abril de 2023}}
  \end{textblock}
  
  % the logos
  \begin{textblock}{12}(1.0,0.3)
      \includegraphics[width=6cm]{LOGO-WSU-wsu_primary_horz_color}
  \end{textblock}
  \begin{textblock}{10}(12.0,0.2)
      \includegraphics[width=2.0cm]{CERN_logo-689x700}
  \end{textblock}
  \begin{textblock}{10}(14.3,0.2)
      \includegraphics[width=1.5cm]{LOGO-ALICE-2012-Jul-04-4_Color_Logo_CB}
  \end{textblock}
  \begin{textblock}{10}(0.2,2.9)
      \includegraphics[width=6.0cm]{jorgeguillenlogo}
  \end{textblock}
\end{frame}

\section{La aventura del conocimiento}

\begin{frame}{a slide}
  \begin{textblock}{16}(0.0,0.0)
    \includegraphics[width=1\textwidth,keepaspectratio=true,
                     clip=true,trim=0pt 0pt 0pt 0pt]
                     {Kheops-Pyramid}
  \end{textblock}
By Nina - Own work, CC BY 2.5, https://commons.wikimedia.org/w/index.php?curid=282496
\end{frame}

\section{\LARGE \textbf{La Física}}

\begin{frame}
  \frametitle{\LARGE \textbf{Contribución de la Física}}
  \begin{textblock}{15}(0.5,1.5)
    \includegraphics[width=0.49\textwidth,keepaspectratio=true,
                     clip=true,trim=20pt 0pt 0pt 0pt]
                     {pedaa5b25f02_hr}
    \includegraphics[width=0.49\textwidth,keepaspectratio=true,
                     clip=true,trim=0pt 100pt 0pt 0pt]
                     {s-l500}
  \end{textblock}

\end{frame}

\section{\LARGE \textbf{La partículas (materia) son ondas}}

\begin{frame}
  
\end{frame}


\section{\LARGE \textbf{¿Se transmite la luz\\instantáneamente?\\¿Cómo se mide\\la velocidad de la luz?}}

\begin{frame}
  \frametitle{\LARGE\textbf{Ole Christensen Rømer(1676)}}
  \begin{textblock}{14}(1.0,1.3)
    \centering
    \includegraphics[width=1\textwidth,keepaspectratio=true,
                     clip=true,trim=0pt 0pt 0pt 0pt]
                     {Io_eclipse_speed_of_light_measurement}\\
    \vspace{-0.1in}
    {\color{blue}\scriptsize By Cmglee - Own work, CC BY-SA 3.0
    \href{https://commons.wikimedia.org/w/index.php?curid=32388964}
    {https://commons.wikimedia.org/w/index.php?curid=32388964}}
  \end{textblock}
\end{frame}


\section{\LARGE \textbf{La ondas son partículas}}



\begin{frame}
  \frametitle{\LARGE \textbf{Relatividad general}}
  \begin{textblock}{7.35}(8.0,1.5)
    \includegraphics[width=1\textwidth,keepaspectratio=true,
                     clip=true,trim=0pt 100pt 0pt 0pt]
                     {s-l500}
  \end{textblock}

  \begin{textblock}{6}(2.5,5.7)
    \includegraphics[width=1\textwidth,keepaspectratio=true,
                     clip=true,trim=0pt 0pt 100pt 30pt]
                     {dark-energy-now}
  \end{textblock}
  \begin{textblock}{7.35}(0.5,1.7)
    \begin{itemize}
      \item {\color{blue}\textbf{Espaciotiempo}}
      \vspace{0.05in}
      \item {\color{blue}\textbf{Dilatación del tiempo}}
      \vspace{0.05in}
      \item {\color{blue}\textbf{Agujeros negros}}
      \vspace{0.05in}
      \item {\color{blue}\textbf{Materia oscura}}
      \vspace{0.05in}
      \item {\color{blue}\textbf{Energía oscura}}
    \end{itemize}
  \end{textblock}
\end{frame}

\begin{frame}{The waves video}
  \begin{textblock}{16.05}(-0.02,-0.02)
  \includemedia[
    width=1\linewidth,
    height=0.565\linewidth,
    keepaspectratio,
    activate=pageopen,
    playbutton=fancy,
    addresource=VIDEOS/waves.mp4,
    flashvars={source=VIDEOS/waves.mp4&autoPlay=true&loop=true}
    ]{}{VPlayer.swf}
  \end{textblock}
\end{frame}

\begin{frame}{The light double slit}
  \begin{textblock}{16.05}(-0.02,-0.02)
  \includemedia[
    width=1\linewidth,
    height=0.565\linewidth,
    keepaspectratio,
    activate=pageopen,
    playbutton=fancy,
    addresource=VIDEOS/lightdslit.mp4,
    flashvars={source=VIDEOS/lightdslit.mp4&autoPlay=true&loop=true}
    ]{}{VPlayer.swf}
  \end{textblock}
\end{frame}

\begin{frame}{The light double slit}
  \begin{textblock}{16.05}(-0.02,-0.02)
  \includemedia[
    width=1\linewidth,
    height=0.565\linewidth,
    keepaspectratio,
    activate=pageopen,
    playbutton=fancy,
    addresource=VIDEOS/ballsdslit.mp4,
    flashvars={source=VIDEOS/ballsdslit.mp4&autoPlay=true&loop=true}
    ]{}{VPlayer.swf}
  \end{textblock}
\end{frame}

\begin{frame}{The light double slit}
  \begin{textblock}{16.05}(-0.02,-0.02)
  \includemedia[
    width=1\linewidth,
    height=0.565\linewidth,
    keepaspectratio,
    activate=pageopen,
    playbutton=fancy,
    addresource=VIDEOS/measuredslit.mp4,
    flashvars={source=VIDEOS/measuredslit.mp4&autoPlay=true&loop=true}
    ]{}{VPlayer.swf}
  \end{textblock}
\end{frame}



\end{document}

\begin{frame}{The CERN video}
  \begin{textblock}{16.05}(-0.02,-0.02)
  \includemedia[
    width=1\linewidth,
    height=0.565\linewidth,
    keepaspectratio,
    activate=pageopen,
    playbutton=fancy,
    addresource=VIDEOS/CERN-VIDEO-2019-019-001-1080p.mp4,
    flashvars={source=VIDEOS/CERN-VIDEO-2019-019-001-1080p.mp4&autoPlay=true}
    ]{}{VPlayer.swf}
  \end{textblock}

\end{frame}


\begin{frame}{The ALICE video}
  \begin{textblock}{16.1}(0.0,0.0)
  \includemedia[
    width=1\linewidth,
    height=0.56\linewidth,
    keepaspectratio,
    activate=pageopen,
    playbutton=fancy,
    addresource=VIDEOS/OPEN-VIDEO-2021-344-001.mp4,
    flashvars={source=VIDEOS/OPEN-VIDEO-2021-344-001.mp4&autoPlay=true}
    ]{}{VPlayer.swf}
  \end{textblock}

\end{frame}


\end{document}
